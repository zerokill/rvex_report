\chapter{Background}
\label{chap:background}
\section{VEX System}
asd
\subsection{Architecture}
asd
\subsection{ISA}
asd
\subsection{Registers}
asd
\subsection{Run-time architecture}
asd

\section{LLVM Compiler infrastructure}
LLVM is based on the classic three-stage compiler architecture shown in figure \ref{fig:compiler_structure}. The compiler uses a number language specific front-ends, an optimizer and target specific backends. This modular design enables compiler designers to work on different parts of the compiler as a separate part. Support for a new processor can be added by building a new back-end. The existing front-end and optimizer can be reused for the new compiler.

\begin{figure}[ph!]
\centering
\includegraphics[width=0.8\textwidth]{2_background/img/Basic_compiler.png}
\caption{Basic compiler structure}
\label{fig:compiler_structure}
\end{figure}

The front-end is used to transform the plain text source code of a program into an intermediate representation that will be used during compilation process. This transformation is achieved by performing the following steps:

\begin{enumerate}
	\item \textbf{Lexical analysis:} Break input into individual tokens.
	\item \textbf{Syntax analysis:} Using a grammer the sequence of tokens is transformed into a parse tree which represents the structure of the program.
	\item \textbf{Semantic analysis:} Semantic information is added to the parse tree, type checking is performed and a symbol table is built.
\end{enumerate}

The resulting abstract syntax tree (AST) is transformed into LLVM IR and passed to the optimizer and backend of the compiler. These parts of the compilation process are completely language agnostic and do not require any other information from the front-end.

The optimizer is used to analyze and optimize the program. Optimization such as dead code elimination and copy propagation are performed during this phase but also more advanced operations that extract ILP (loop vectorization) can be enabled.

The back-end optimizes and generates code for a specific architecture. The LLVM IR is transformed into processor specific assembly instructions, allocates registers and schedules code for better performance.

\subsection{Front-end}
The modular design of LLVM enables the compiler to be used as a part of the existing GCC compiler. For example, the dragonegg GCC plugin is designed to replace the GCC code generator and optimizer with the LLVM backend. This would enable LLVM to be able to use the existing GCC based front-ends and supported languages.

Clang has been developed to allow LLVM to operate independently of GCC. Clang is a front-end supporting C, C++ and ObjC. The front-end is designed to be closely integrated with the Integrated Development Environment (IDE) allowing more expressive diagnostic messages. In addition Clang also aims to provide faster compilation and lower memory usage \cite{clang:features}.

\subsection{LLVM IR}
The front-end transforms a source code into the LLVM internal representation (LLVM IR). The LLVM IR is used to represent a high level language cleanly in a target independent way and is used during all phases of compilation. Instructions are similar to RISC instructions and can use three operands. Control flow instructions and type specific load/store instructions are used and an infinite amount of registers are available in Single Static Assignment (SSA) form. The LLVM IR is available as human readable text, in memory and in binary form \cite{llvm:presentation}.

The LLVM IR is designed to expose high-level information for further optimization. Examples of high-level information include dataflow through SSA form, control-flow graphs, language independent type information and explicit use of pointer arithmetic. 

Primitives such as voids, floats and integers are natively supported in the LLVM IR. The bitwidth of the integers can be defined manually. Pointers, arrays, structures and functions are derived from these basic types. The operations that are supported in LLVM IR are contained in the Instruction Selection DAG (ISD) namespace.

Object oriented constructs such as classes and virtual methods are not natively supported but can be built using the existing type system. For example a C++ class can be represented by a struct and a list of methods. 

The SSA based dataflow form allows the compiler to efficiently perform code optimizations such as dead code elimination and constant propagation. 

Figure \ref{lst:C_Example} shows an example program in C. The equivalent LLVM IR representation is shown in figure \ref{lst:LLVM_IR}.

\lstset{numbers=none, captionpos=b}
\begin{lstlisting}[language=C,caption={C example program},label=lst:C_Example]
int main() {
	int sum = 1;

	while(sum < 10)
	{
		sum = sum + 1;
	}
	return sum;
}
\end{lstlisting}


\lstset{numbers=none, captionpos=b}
\begin{lstlisting}[language=llvm,caption={LLVM Intermediate representation},label=lst:LLVM_IR]

define i32 @main() nounwind ssp uwtable {
  %1 = alloca i32, align 4
  %sum = alloca i32, align 4
  store i32 0, i32* %1
  store i32 1, i32* %sum, align 4
  br label %2

; <label>:2                                       ; preds = %5, %0
  %3 = load i32* %sum, align 4
  %4 = icmp slt i32 %3, 10
  br i1 %4, label %5, label %8

; <label>:5                                       ; preds = %2
  %6 = load i32* %sum, align 4
  %7 = add nsw i32 %6, 1
  store i32 %7, i32* %sum, align 4
  br label %2

; <label>:8                                       ; preds = %2
  %9 = load i32* %sum, align 4
  ret i32 %9
}

\end{lstlisting}

\subsection{Code generation}
During code generation the optimized LLVM IR is translated into machine specific assembly instructions. The modular design of LLVM enables generic algorithms to be used for this process. 

A backend is described in a domain specific language (DSL) called tablegen. The tablegen files describe properties of a backend such as available instructions, registers, calling convention and pipeline structure. During compilation of LLVM the tablegen files are converted into a C++ description of the backend. Tablegen has been specifically designed to describe the backend structure in a flexible and generic way. Common features can be more easily described using tablegen. For example the $Add$ and $Sub$ instruction are almost identical and using tablegen can be described in a more generic way. This results in less repetition and reduces the chance of error.

Because of the generic description of the backend large amount of code can be reused by each backend. Algorithms such as register allocation and instruction selection operate on the generic tablegen descriptions and do not require target specific hooks to operate correctly. An additional advantage of this approach is that multiple algorithms are available to achieve certain functionality. For example, LLVM offers the developer a choice between four different register allocation algorithms. Each algorithm has a number of advantages and disadvantages and the developer can choose between an algorithm which matches the target processor best.

At the moment not all parts of the backend can be described in Tablegen and hand written C++ code is still needed. As LLVM develops more parts of the backend description should be integrated into the backend. 

Figure \ref{fig:codegen_process} shows the basic codegeneration process. Each block can consist of multiple LLVM passes. For example the instruction selection phase consists of multiple passes that transform the input LLVM IR into a DAG that only contains instructions and types that are supported by the target processor.
\begin{figure}[ph!]
\centering
\includegraphics[width=0.8\textwidth]{2_background/img/Codegen.png}
\caption{Basic codegeneration process}
\label{fig:codegen_process}
\end{figure}

\subsection{Scheduling} % (fold)
\label{sub:scheduling}
The LLVM compiler uses basic blocks to schedule instructions. A basic block is a block of code which has exactly one entry point and one exit point. This means that no jump instruction exists with a destionation in the block. 

LLVM uses the 'MachineBasicBlock' (MBB) class to represent a Basic Block. A MBB contains a list of 'MachineInstr' instances. The 'MachineInstr' class is an abstract way to represent instructions for the target processor.

Multiple MBB are used to create a 'MachineFunction' instance. The 'MachineFunction' class is used to represent a LLVM IR function. In addition to a list of MBB the MachineFunction also contains references to the 'MachineConstantPool', 'MachineFrameInfo', 'MachineFunctionInfo' and 'MachineRegisterInfo'. These classes keep track of target specific function information and are used to store which constants are spilled to memory, the objects that are allocated on the stack, target specific function information and which registers are used respectively. 

% subsection scheduling (end)


\section{Verification}
Verification of the LLVM compiler is extremely important. Performance of the generated binaries is irrelevant if only half of the binaries produce a correct result. The LLVM compiler will be verified by simulating the generated binaries. 

Two simulators are available; XSTsim and Modelsim. XSTsim is an ISA simulator that can simulate a 4 issue width $\rho$-VEX processor. Output of the simulator is customizable and the simulator is fast enough to simulate large executables. Modelsim is used to perform a complete functional simulation of the $\rho$-VEX processor. The Modelsim simulation provides the highest accuracy because an actual $\rho$-VEX processor is simulated and is used for execution. The disadvantage of using Modelsim is the performance. Simulation of an executable will take a long time because the complete processor is simulated.

Verification of the compiler will be performed by writing test programs that generate certain instruction sequences. The output of these test programs will be compared to the expected result to check for errors in the backend.

Writing testprograms that have a high coverage of all the possible output patterns is impossible because of this a second round of verification will be performed by compiling benchmark programs that execute common programs. The output of the benchmarking programs will be compared to expected outputs to check for errors in the compiler.

\section{Conclusion}
This chapter discussed the background of the $\rho$-VEX processor and the basics of the LLVM compiler framework. Instructions that are supported by the $\rho$-VEX processor have been shown, the run-time architecture has been discussed and the register layout of the $\rho$-VEX processor has been discussed. The background information on the LLVM compiler should provide for a basic understanding on how the compiler operates and what parts need to be implemented to build a $\rho$-VEX backend.

Finally we have also discussd how the LLVM compiler will be verified. The verification step is extremely important to determine wheter the binaries that are generated work correctly.

\acresetall
