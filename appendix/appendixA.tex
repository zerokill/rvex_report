\chapter{LLVM Quickstart guide}
In this section we will describe how the LLVM-based compiler can be used for projects. This guide assumes that \emph{clang} and\emph{vex\_llv} executables are available. Further we also require the availability of the VEX assembler and VEX linker.

All the source code that is required for this guide is contained in the \texttt{source.zip} archive. This archive contains the following folders:

\begin{itemize}
	\item \textbf{config:} Configuration files for vex\_llc. Different configuration files exist for 2-issue, 4-issue, and 8-issue machines.
	\item \textbf{benchmarks:} Folder containing the powerstone benchmark suite.
	\item \textbf{tests:} Folder containing the LLVM verification tests.
	\item \textbf{float:} Folder containing the floating point library used during compilation.
\end{itemize}

Each folder contains a \texttt{Makefile} that automates the build process. The current version of the \texttt{Makefiles} are used to compile \texttt{c} sourcecode into $\rho$-VEX assembler.

\section{Compilation}
The following code uses \emph{clang} to compile input \texttt{c} sourcecode into LLVM IR code.
\begin{lstlisting}

	clang -emit-llvm -S -m32 -O0 -fno-stack-protector input.c -o output.ll

\end{lstlisting}

The compiler flags are used as follows:

\begin{itemize}
	\item \texttt{-emit-llvm:} Emits the LLVM IR code.
	\item \texttt{-s:} Emit human readable assembler.
	\item \texttt{-m32:} \texttt{int} is 32-bits.
	\item \texttt{-O0:} No optimizations.
	\item \texttt{-fno-stack-protector:} Do not emit a stack-protector. The current run-time library does not support stackprotection.
\end{itemize}

\emph{vex\_llc} is used as follows:

\begin{lstlisting}

	vex_llc input.ll -march=rvex -mcpu=rvex-vliw -config=../config/rvex_W4_2 -enable-misched=true -relocation-model=static -o output.s

\end{lstlisting}

The compiler flags are used as follows:

\begin{itemize}
	\item \texttt{-march:} Select the architecture to compile for.
	\item \texttt{-mcpu:} Select the subtarget of the architecture.
	\item \texttt{-config:} Path to configuration file.
	\item \texttt{-enable-misched:} Enables the machine scheduler that is required for correct scheduling of $\rho$-VEX operations.
	\item \texttt{-relocation-model:} Current version of $\rho$-VEX processor has no support for dynamic binaries.
\end{itemize}

A executable can be generated using \emph{rvex-as} and \emph{rvex-ld} as follows:

\begin{lstlisting}

	rvex-as --issue 4 --borrow 1,3.0,2.3,1.2,0. --config 9335 -h input.s -o output.o
	rvex-ld  output.o _start.o -o output

\end{lstlisting}

The assembler flags are used as follows:

\begin{itemize}
	\item \texttt{-issue:} Select issue width of final binary.
	\item \texttt{-borrow:} Borrowing scheme used for large immediate values.
	\item \texttt{-config:} Describes which functional units support Multiply, Branch and Load instructions.
	\item \texttt{-h:} Flag toggles generic binaries.
\end{itemize}

The linker is used to build the final executable using a startup file and all the object files.

\section{Simulation}
Simulation can be performed using XSTsim or with Modelsim. XSTsim can 

\begin{lstlisting}

	xstsim-r-VEX-1.1.3 --ips='"[r-VEX c]"'  --c.trace=5 --c.trace_regs=2 --accuracy=0 --c.target_exec='"test"'

\end{lstlisting}

The flags that are used for xstsim are described in the xstsim documentation.

For simulation with modelsim \texttt{hex} files are required. These can be generated with the \emph{rvex-elf2vhd} tool as follows:

\begin{lstlisting}
	rvex-elf2vhd --hex test
\end{lstlisting}

The \emph{elf2vhd} tool can be used to generate \texttt{hex} and \texttt{vhd} files with the following flags.

\begin{itemize}
	\item \texttt{-hex:} Generates \texttt{hex} files.
	\item \texttt{-vhd:} Generates \texttt{vhd} files.
\end{itemize}

The \texttt{vhd} can only be used for 4-issue width $\rho$-VEX processors because the instruction memory datatype is fixed to 128-bits. The \texttt{hex} files can be used for all issue widths an can also be used to load the instruction and data memory of physical hardware.

