\chapter{Introduction}
\section{Motivation}
% \subsection{Origin and history}

In 2008 Thijs van As designed the first version of the $\rho$-VEX processor \cite{As:2008rt}. This processor uses a VLIW design and is based on the VEX ISA. The VEX ISA is a derivative of the Lx family of embedded VLIW processors \cite{854391} from HP/STMicroelectronics.

Around this processor a set of tools has been developed in collaboration with the TU Delft, IBM, STMicroelectronics and other universities. Currently the $\rho$-VEX 2.0 tool suite include a synthesizable core, a compiler system and a processor simulator. A GCC based VLIW compiler has been developed by IBM. 

The origins of the VEX ISA can be traced to the company Multiflow and John Fisher, one of the inventors of VLIW processors at Yale University \cite{Fisher:1983:VLI:1067651.801649}. Multiflow designed a computer that used VLIW processors to execute instructions up to 1024-bit in size. Along with these computers Multiflow also designed a compiler system that used trace based scheduling to extract ILP from programs. Reportedly the code base for the Multiflow compiler has been used in modern compiler such as Intel C Compiler (ICC) and HP VEX compiler because of the robustness and the amount of ILP that could be exposed by the compiler \cite{Lowney:1993qy}.


\subsection{Futore of $\rho$-VEX}
asd
\section{Problem statement}
asd
\subsection{Goals}
asd
\subsection{Subgoals}
asd
\section{Methodology}
asd
\section{Thesis overview}
asd


\acresetall
