\chapter{Introduction}
\section{Motivation}
% \subsection{Origin and history}

In 2008 Thijs van As designed the first version of the $\rho$-VEX processor \cite{As:2008rt}. This processor uses a VLIW design and is based on the VEX ISA. The VEX ISA is a derivative of the Lx family of embedded VLIW processors \cite{854391} from HP/STMicroelectronics.

Around this processor a set of tools has been developed in collaboration with the TU Delft, IBM, STMicroelectronics and other universities. Currently the $\rho$-VEX 2.0 tool suite include a synthesizable core, a compiler system and a processor simulator. A GCC based VLIW compiler has been developed by IBM. 

VLIW differs from multiple issue in that parallelism is found during compile-time instead of during run-time. This results in a processor that can be made significantly simpler because OoO does not need to be implemented. The compiler is responsible for finding all the parallelism and for scheduling code in an efficient way. This also enables the compiler to execute additional optimization because the compiler has got a higher level view of the code that needs to be executed. Optimizations such as \emph{swing modulo scheduling} and loop vectorization are nearly impossible to achieve in hardware because the higher level structure is no longer available. A compiler can interpret the higher level structure of a program and optimize the output for better scheduling. 

The origins of the VEX ISA can be traced to the company Multiflow and John Fisher, one of the inventors of VLIW processors at Yale University \cite{Fisher:1983:VLI:1067651.801649}. Multiflow designed a computer that used VLIW processors to execute instructions up to 1024-bit in size. Along with these computers Multiflow also designed a compiler system that used trace based scheduling to extract ILP from programs. Reportedly the code base for the Multiflow compiler has been used in modern compiler such as Intel C Compiler (ICC) and HP VEX compiler because of the robustness and the amount of ILP that could be exposed by the compiler \cite{Lowney:1993qy}.


\subsection{Futore of $\rho$-VEX}
asd
\begin{itemize}
	\item Runtime reconfiguration of processor and of compiler.
	\item Possible JITing of code
	\item Generic binaries, one size fits all
\end{itemize}
\section{Problem statement}
Currently both HP-VEX and GCC can be used to generate code for the rvex processor. Both compilers have got a number of advantages and disadvantages that will be explored. The compilers will be judged on the following subjects: Code quality, support, languages support, backend supported and customization possibilities.

% \begin{itemize}
% 	\item \textbf{Code quality:}
% 	\item \textbf{Support:}
% 	\item \textbf{Front-end:}
% 	\item \textbf{Back-end:}
% 	\item \textbf{Customization:}
% \end{itemize}
HP-VEX:
\begin{itemize}
	\item \textbf{Code quality:} Excellent code quality and ILP extraction.
	\item \textbf{Support:} Bad, no active community.
	\item \textbf{Front-end:} Bad, only support for C.
	\item \textbf{Back-end:} Not applicable since compiler is specifically targeted to one architecture.
	\item \textbf{Customization:} Customization possible through machine description. Further research on optimization strategies not possible because compiler is proprietary and closed source. Because of this expanding the functionality of the compiler is impossible.
\end{itemize}

GCC:
\begin{itemize}
	\item \textbf{Code quality:} Excellent code quality with performance approaching that of commercial compilers (CITATION NEEDED).
	\item \textbf{Support:} There is a very active development community around GCC.
	\item \textbf{Front-end:} GCC supports a large number of programming languages including C, C++, Fortran and Java
	\item \textbf{Back-end:} Supoort exists for a large number of processors including x86, ARM, MIPS and ofcourse VEX
	\item \textbf{Customization:} Because GCC is open source the compiler can be customized to support new passes, optimizations and instructions.
\end{itemize}

Unfortunately GCC has a number of disadvantages that need mentioning.

\begin{itemize}
	\item \textbf{VEX code quality:} The VEX backend for GCC has not been optimized and the quality of the code is quite low. Performance of GCC executables is lower then code compiled by the HP-VEX compiler. Some programs do not function correclty when compiled by GCC. Some programs are unable to be compiled by GCC.
	\item \textbf{VEX reconfiguration:} The current GCC VEX compiler does not support run-time reconfiguration. The compiler has been set to a 4 issue width $\rho$-VEX and this cannot be changed without rebuilding GCC.	
	\item \textbf{Bloated:} GCC consists of millions of lines of code and is arguable one of the most complex programs in existence. This makes understanding GCC and developing for GCC very hard.
	\item \textbf{Complexity:} GCC is written in C. Design is complex, not very modular and documentation is not very good. Different parts of the compiler are linked in a complex way and it is very difficult to obtain a general picture on how the compiler operates. Because of the complexity it is difficult to achieve high performance in GCC.
\end{itemize}

The comparison shows that both the HP-VEX and GCC compilers have serious disadvantages. The fact that HP-VEX cannot be customized excludes it from further development for the $\rho$-VEX project. Bringing the GCC compiler performance and features up to the same level as HP-VEX will be very difficult because of the complexity involved with GCC development.  

In 2000 the LLVM project has been started with the goal of replacing the code generator in the GCC compiler. LLVM provides a modern, modular design and is written in C++. The compiler 

\subsection{Goals}
asd
\subsection{Subgoals}
asd
\section{Methodology}
asd
\section{Thesis overview}
asd


\acresetall
