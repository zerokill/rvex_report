\chapter{Conclusion}
\label{chap:conclusion}

\section{Summary}
In chapter \ref{chap:background} we discussed the background of the $\rho$-VEX processor and the basics of the LLVM compiler framework. The $\rho$-VEX processor is a VLIW type processor that uses RISC like instructions to operate. We have presented the basic design of the processor, instructions that are supported, register properties, and the run-time architecture has been discussed. This information will be used during implementation of the LLVM-based compiler.

We have also discussed the basic working of the LLVM compiler framework. Building a $\rho$-VEX backend for the LLVM compiler is feasible because the current version of the LLVM compiler already targets a VLIW processor. 

Finally we have also discussd how the LLVM compiler will be verified. The verification step is extremely important to determine wheter the binaries that are generated work correctly.

Chapter \ref{chap:implementation} discussed how the LLVM-based compiler has been implemented. Using the \texttt{tablegen} description we have described all the instructions that are supported by the $\rho$-VEX processor. We have also shown how LLVM IR code is transformed into a DAG that represents the original program. Through lowering functions this DAG is transformed into a DAG that contains only operations that are supported by the target processor. With instruction selection the DAG is transformed into a new DAG that contains target specific operations.

The finished DAG is transformed into a sequential list of instructions. This instruction list is used for the remaining passes. The remaining passes are used to map virtual registers to physical registers, emit prologue and epilogue functions and to perform VLIW packetization.

Certain features that are required for the $\rho$-VEX processor are not available in the LLVM compiler. Features such as a machine description file are new to LLVM and we have shown what changes have been made to the LLVM compiler to support machine description files. In addition, we have shown how the backend has been updated to provide support for the $\rho$-VEX generic binary format. Support for floating point operations has been added by porting the LLVM floating point library to the $\rho$-VEX processor.

Chapter \ref{chap:optimization} discussed the optimizations that have been implemented to improve performance of the $\rho$-VEX binaries. The \texttt{rvexMachineScheduler} pass is used to handle structural and data hazards. Temporary instruction packets are generated and are filled with instructions that have been selected using cost based scheduling algorithms to reduce register pressure. The pass enables $\rho$-VEX binaries to execute correclty and to perform better then binaries that have not been scheduled using the \texttt{rvexMachineScheduler}.

The branch analysis optimization is used to erase unnecessary \texttt{goto} statements from the code. In addition hooks have been provided that allow the LLVM \texttt{BranchFolding} pass to further optimize branches that are used in $\rho$-VEX binaries.

The generic binary optimization allows binaries with generic binary support to perform on par with regular binaries. The performance of generic binaries will only degrade once the register pressure becomes too high and spill code needs to be inserted.

The immediate value optimization allows more efficient use of available instructions of the  $\rho$-VEX processor.

Finally in chapter \ref{chap:results} we have shown how the operation of the LLVM-based compiler has been verified and how well binaries execute that have been generated with the LLVM-based compiler.


The benchmarks and verifications have shown that the LLVM-based compiler still contains bugs. Not all benchmarks are able to execute using the Modelsim simulator but all benchmarks are able to execute using the XSTsim simulator. This indicates that there are probably scheduling issues in the assembly that is generated. The reason for the failing benchmarks remain unclear. It is possible that the compiler generates code that is not scheduled properly but analysis of the \texttt{jpeg} benchmark indicated the possibility of other issues. 

We have shown that the LLVM-based compiler exceeds the performance of the GCC-based compiler but the compiler is still outperformed by the HP-VEX compiler. As expected the HP-VEX compiler generates binaries that perform very well. This is related to the trace based scheduling techniques that are employed to extract a high level of ILP \cite{Lowney:1993qy}. In addition, the HP-VEX compiler also performs certain optimizations that are not available to the LLVM-based compiler at \texttt{-O0}.

Additionally, the benchmarks have also shown that the LLVM-based compiler is the only compiler able to generate correct code for all selected benchmarks. Surprisingly, even the HP-VEX compiler generates incorrect binaries for certain benchmarks. The code quality of the GCC-based compiler is bad with four benchmarks failing to execute.

Furthermore, we have also shown that the generic binary optimization allows generic binaries to operate at speeds that are nearly equal to the regular binaries. The generic binary optimization does introduce spill code in benchmarks that already use a large number of physical registers. The current optimization is too aggressive for the \texttt{adpcm} benchmark and introduces excessive amount of spill code that degrades performance. The optimization should be fine tuned to consider this situation. More problematic is the fact that the used optimization breaks certain benchmarks.

\section{Main contributions}
In this thesis we have presented the design of a LLVM-based $\rho$-VEX compiler. We have shown how the compiler is built, what optimizations have been implemented and the performance of binaries that have been generated with the LLVM-based compiler. Not all benchmarks are able to execute properly which is related to the scheduling constraints imposed by the $\rho$-VEX processor.

The following parts have been contributed to the $\rho$-VEX project:

\begin{itemize}
	\item LLVM-based $\rho$-VEX compiler.
	\begin{itemize}
		\item Floating point support.
		\item 64-bit integer support
	\end{itemize}
	\item Parameterization of LLVM-based compilers.
	\item Support for generic binaries.
	\item Scheduling optimizations that improve performance of regular binaries.
	\item Register allocation optimization that improves performance of generic binaries
\end{itemize}

\section{Future work}
Future work for the LLVM-based $\rho$-VEX could involve the following subjects.

\begin{itemize}
	\item \textbf{Fix compilation bugs:}  The current version of the LLVM-based compiler shows possible erors in a number of benchmarks. These bugs need to be fixed to produce a more stable version of the LLVM-based compiler

	\item \textbf{Optimization support:} Currently we have not considered using higher optimization levels. Brief tests with higher optimization levels indicated massive increases in performance. Unfortunetely, the compiler started generating instruction patterns that are not yet supported by the $\rho$-VEX backend.

	\item \textbf{Optimize scheduling:} The scheduling algorithms that are used by the LLVM-based compiler can be further optimized. In \cite{Vahedi:2013mz} a technique is presented to optimize the scheduling for VLIW processors. The algorithm uses information about the instruction delay to reorder the instructions. This leads to an increase in pipeline utilization and code that performs better. Scheduling could also be improved by integrating the register allocation and instruction scheduling phase as discussed in \cite{Bradlee:1991:IRA:106973.106986}. 

	\item \textbf{Enhance parameterization:} At the moment only issue-width, instruction stages and instruction delay can be customized through config files. In the future more configuration options can be added such as number of registers available or scheduling parameters.

	\item \textbf{LLVM JIT:} The LLVM compiler support Just-In-Time compilation through the LLVM interpreter. Implementing the interpreter for the $\rho$-VEX processor could produce interesting results where binary properties can be modified during runtime. For example the interpreter could look for code with higher degree of ILP. If suitable code is found the program will be executed on an 8-issue $\rho$-VEX processor. If suitable code is not found the issue-width could be reduced to 2- or 4-issue code and the idle functional units can be shut down to preserve energy.

	\item \textbf{Trace based scheduling:} Currently the LLVM compiler uses a basic block scheduler. Introducing a trace based scheduler could further improve performance of binaries for VLIW type processors.

\end{itemize}

\acresetall